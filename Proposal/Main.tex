\documentclass{article}
\usepackage[utf8]{inputenc}

\title{\LARGE \bf
Environmental sound classification
}
\author{}

\begin{document}
\maketitle

%%%%%%%%%%%%%%%%%%%%%%%%%%%%%%%%%%%%%%%%%%%%%%%%%%%%%%%%%%%%%%%%%%%%%%%%%%%%%%%%
\begin{abstract}
Environmental sound classification can be applied to many daily life scenarios, and can be used to identify abnormal events nearby.So we proposed a framework to classify these environmental sounds.In signal Processing, Mel-Frequency Cepstrum (MFC) is a representation of the short-term power spectrum of a sound, Mel-frequency cepstral coefficients (MFCCs) are coefficients that collectively make up an MFC. MFCCs can usually be applied to extract sound features.The ability of deep convolutional neural networks (CNN) to learn spectrol patterns makes them well suited to environmental sound classification.We trained the CNN model and then classified it.
\end{abstract}
%%%%%%%%%%%%%%%%%%%%%%%%%%%%%%%%%%%%%%%%%%%%%%%%%%%%%%%%%%%%%%%%%%%%%%%%%%%%%%%%
\section{Introduction}

\end{document}
